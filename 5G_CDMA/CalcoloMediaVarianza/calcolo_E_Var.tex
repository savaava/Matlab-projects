\documentclass[12pt, a4paper]{article}

% ******************************** PACKAGE ****************************
\usepackage[top=1.5cm, bottom=1.5cm, left=1.5cm, right=1.5cm]{geometry}
\usepackage{amsmath}
\usepackage{amsmath}  % Per espressioni matematiche
\usepackage{amssymb}  % Per simboli matematici
\usepackage{enumitem}
\usepackage{fancyhdr}
%\usepackage{titlesec}  % Pacchetto per personalizzare le sezioni

% ******************************** SETUP ********************************
\pagestyle{fancy} % Attiva lo stile fancy
\fancyhf{} % Pulisce intestazione e piè di pagina
\fancyfoot[r]{\textit{\textbf{Andrea Savastano}}} % Piè di pagina centrato

% ******************************** DOCUMENTO ****************************
\begin{document}
	
\begin{center} \section*{Ricezione con CDMA} \end{center}

\begin{itemize}[itemsep=0pt]
	\item $N$: numero di utenti
	\item $\mathcal{E}_s$: Energia del segnale trasmesso $s$ 
	\item $s_1$: segnale vettoriale aspettato 
	\item $c_n$: chirping code dell'utente \textit{n}-esimo
	\item $L_c$: lunghezza del chirping code
\end{itemize}

\[
s_1 = \mathcal{E}_s \pm \sum_{k=1}^{L_c}\left(\sum_{n=2}^{N}c_{1k}\cdot c_{nk}\right)
  = \mathcal{E}_s \pm \sum_{k=1}^{L_c}X_k
\] 

\begin{itemize}[itemsep=0pt]
	\item $\{X_k\}_{k=1,\dots,L_c}$: variabili aleatorie indipendenti
\end{itemize}
\hrule

\section{Calcolo di $\mathrm{E}[X_k]$ e $\mathrm{VAR}[X_k]$}
\[
\Bigg( c_{nk}\in [-1,1] \Bigg) \implies \Bigg( \mathrm{E}[c_{nk}] = 0 \Bigg)
\] \\\vspace{-.5cm} \[
\mathrm{\mathrm{E}}[X_k] = \mathrm{E}\left[\sum_{n=2}^{N}c_{1k}\cdot c_{nk}\right] 
       = c_{1k}\cdot \mathrm{E}\left[\sum_{n=2}^{N}c_{nk}\right]
       = c_{1k}\cdot \sum_{n=2}^{N}\mathrm{E}[c_{nk}] 
       = 0
\] \\\vspace{-.5cm} \[
\mathrm{VAR}[X_k] = \mathrm{E}[X_k^2] - \mathrm{E}^2[X_k] = \mathrm{E}[X_k^2] = \mathrm{E}\left[\left(\sum_{n=2}^{N}c_{1k}\cdot c_{nk}\right)^2\right] =
\mathrm{E}\left[ c_{1k}^2\cdot \sum_{n=2}^{N}c_{nk}^2 \right] =
\mathrm{E}\left[ \sum_{n=2}^{N}1 \right] =
N-1
\]
\hrule
	
\section{Calcolo di $\mathrm{E}[n]$ e $\mathrm{VAR}[n]$}
\[
\mathrm{E}[n] = \mathrm{E}\left[ \sum_{k=1}^{L_c}X_k \right] =
\sum_{k=1}^{L_c}\mathrm{E}\left[ X_k \right] = \sum_{k=1}^{L_c}0 = 0
\] \\\vspace{-1.3cm} 

\[
\Bigg( \{X_k\}_{k=1,\dots,L_c}	\text{indipendenti} \Bigg) \implies \Bigg(
\mathrm{VAR}[X_1+X_2+\dots+X_{Lc}] = \mathrm{VAR}[X_1]+\mathrm{VAR}[X_2]+\dots+\mathrm{VAR}[X_{Lc}] \Bigg)
\] \\\vspace{-1.2cm} 

\[
\implies \mathrm{VAR}[n] = \mathrm{VAR}\left[ \sum_{k=1}^{L_c}X_k \right] =
\sum_{k=1}^{L_c}\mathrm{VAR}\left[ X_k \right] = 
\sum_{k=1}^{L_c}\left( N-1 \right) = 
L_c \cdot (N-1)
\] \\
Applicando il \textit{Teorema Centrale del Limite},\\
essendo $\{X_k\}_{k=1,\dots,L_c}$
indipendenti con $\mathrm{E}[X_k]=0$ e $\mathrm{VAR}[X_k]=(N-1)$ la loro somma genera una variabile aleatoria $n$ con distribuzione Gaussiana e questa è il rumore che si aggiunge in ricezione.
\[
\sum_{k=1}^{L_c}X_k = n, \quad\boxed{ n\sim\mathcal{N}\Big(0,\, L_c(N-1)\Big) }
\implies \Bigg( s_1 = \mathcal{E}_s \pm \sum_{k=1}^{L_c}X_k = \mathcal{E}_s \pm n \Bigg)
\]

\end{document}